\subsection*{High Throughput Computing (HTC)}
For many experimental scientists, scientific progress and quality of research are strongly linked to computing throughput. 

In other words, most scientists are concerned with how many floating point operations per month or per year they can extract 
their computing environment rather than the number of such operations the environment can provide them per second or minute. 
Floating point operations per second (FLOPS) has been the yardstick used by most High Performance Computing (HPC) efforts to 
evaluate their systems. 

Little attention has been devoted by the computing community to environments that can deliver large 
amounts of processing capacity over long periods of time. We refer to such environments as High Throughput Computing (HTC) 
environments.

\subsection{Key Definitions}
For more than a decade, the HTCondor team at the Computer Sciences Department at the University of Wisconsin-Madison has been 
developing and evaluating mechanisms and policies that support HTC on large collections of distributively owned heterogeneous 
computing resources. We first introduced the distinction between High Performance Computing (HPC) and High Throughput Computing 
(HTC) in a seminar at the NASA Goddard Flight Center i in July of 1996 and a month later at the European Laboratory for 
Particle Physics (CERN). In June of 1997 HPCWire published an interview on High Throughput Computing.

The key to HTC is effective management and exploitation of all available computing resources. 

Since the computing needs of most scientists can be satisfied these days by commodity CPUs and memory, high efficiency is 
not playing a major role in a HTC environment. The main challenge a typical HTC environment faces is how to maximize the 
amount of resources accessible to its customers. 

Distributed ownership of computing resources is the major obstacle such 
an environment has to overcome in order to expand the pool of resources it can draw from. 

Recent trends in the cost/performance  ratio of computer hardware have placed the control (ownership) over powerful 
computing resources in the hands of individuals and small groups. 

These distributed owners will be willing to include their resources in a HTC
environment only after they are convinced that their needs will be addressed and their rights protected.
