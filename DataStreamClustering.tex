\subsection*{Data stream clustering}
In computer science, data stream clustering is defined as the clustering of data that arrive continuously such as telephone records, multimedia data, financial transactions etc. Data stream clustering is usually studied as a streaming algorithm and the objective is, given a sequence of points, to construct a good clustering of the stream, using a small amount of memory and time.

Data stream clustering has recently attracted attention for emerging applications that involve large amounts of streaming data. For clustering, k-means is a widely used heuristic but alternate algorithms have also been developed such as k-medoids, CURE and the popular BIRCH.
For data streams, one of the first results appeared in 1980, but the model was formalized in 1998.

%----------------------------------------------------------------------------%
streaming algorithms are algorithms for processing data streams in which the input is presented as a sequence of items and can be examined in only a few passes (typically just one). These algorithms have limited memory available to them (much less than the input size) and also limited processing time per item.

These constraints may mean that an algorithm produces an approximate answer based on a summary or "sketch" of the data stream in memory.

%----------------------------------------------------------------------------%

In data mining, k-means++[1][2] is an algorithm for choosing the initial values (or "seeds") for the k-means clustering algorithm. It was proposed in 2007 by David Arthur and Sergei Vassilvitskii, as an approximation algorithm for the NP-hard k-means problem—a way of avoiding the sometimes poor clusterings found by the standard k-means algorithm. It is similar to the first of three seeding methods proposed, in independent work, in 2006[3] by Rafail Ostrovsky, Yuval Rabani, Leonard Schulman and Chaitanya Swamy. (The distribution of the first seed is different.)
