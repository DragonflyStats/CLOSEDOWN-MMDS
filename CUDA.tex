\subsection*{Compute Unified Device Architecture}

CUDA (Compute Unified Device Architecture) is a parallel computing platform and programming model created by NVIDIA and implemented by the graphics processing units (GPUs) that they produce.[1] CUDA gives program developers direct access to the virtual instruction set and memory of the parallel computational elements in CUDA GPUs.
 
Using CUDA, the GPUs can be used for general purpose processing (i.e., not exclusively graphics); this approach is known as GPGPU. Unlike CPUs, however, GPUs have a parallel throughput architecture that emphasizes executing many concurrent threads slowly, rather than executing a single thread very quickly.
 
The CUDA platform is accessible to software developers through CUDA-accelerated libraries, compiler directives (such as OpenACC), and extensions to industry-standard programming languages, including C, C++ and Fortran. C/C++ programmers use 'CUDA C/C++', compiled with "nvcc", NVIDIA's LLVM-based C/C++ compiler.[2] Fortran programmers can use 'CUDA Fortran', compiled with the PGI CUDA Fortran compiler from The Portland Group.
 
In addition to libraries, compiler directives, CUDA C/C++ and CUDA Fortran, the CUDA platform supports other computational interfaces, including the Khronos Group's OpenCL,[3] Microsoft's DirectCompute, and C++ AMP.[4] Third party wrappers are also available for Python, Perl, Fortran, Java, Ruby, Lua, Haskell, MATLAB, IDL, and native support in Mathematica.
 
In the computer game industry, GPUs are used not only for graphics rendering but also in game physics calculations (physical effects such as debris, smoke, fire, fluids); examples include PhysX and Bullet. CUDA has also been used to accelerate non-graphical applications in computational biology, cryptography and other fields by an order of magnitude or more.[5][6][7][8][9]
 
CUDA provides both a low level API and a higher level API. The initial CUDA SDK was made public on 15 February 2007, for Microsoft Windows and Linux. Mac OS X support was later added in version 2.0,[10] which supersedes the beta released February 14, 2008.[11] CUDA works with all Nvidia GPUs from the G8x series onwards, including GeForce, Quadro and the Tesla line. CUDA is compatible with most standard operating systems. Nvidia states that programs developed for the G8x series will also work without modification on all future Nvidia video cards, due to binary compatibility.[citation needed]

%--------------------------------------------------------------------------------%
\end{document}
