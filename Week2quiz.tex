Week2
This page features MathJax technology to render mathematical formulae. If you are using a screen reader, please visit MathPlayer to download the plugin for your browser. Please note that this is an Internet Explorer-only plugin at this time.Mining Massive Datasets
Top Navigation BarCoursesKevin O'Brien
stanford
Mining Massive Datasets
by Jure Leskovec, Anand Rajaraman, Jeff Ullman
Course Home Page
Search this course Search
Side Navigation Bar
COURSE

Announcements
Video Lectures
Video Errata
EXERCISES

Quizzes (Homeworks)
Surveys
ABOUT THE COURSE

Syllabus
Grading and Logistics
COMMUNITY

Discussion Forums
Course Wikiopens in new browser tab
Join a Meetupopens in new browser tab
Help Articles
Course Materials Errors
Technical Issues
Week 2: LSH (Basic) Help

The due date for this quiz is Mon 20 Oct 2014 11:59 PM PDT.

In accordance with the Coursera Honor Code, I (Kevin O'Brien) certify that the answers here are my own work.
Question 1
The edit distance is the minimum number of character insertions and character deletions required to turn one string into another. Compute the edit distance between each pair of the strings he, she, his, and hers. Then, identify which of the following is a true statement about the number of pairs at a certain edit distance.
There are 4 pairs at distance 5.
There are 4 pairs at distance 2.
There is 1 pair at distance 4.
There are 4 pairs at distance 1.
Question 2
Consider the following matrix:
C1	C2	C3	C4
R1	0	1	1	0
R2	1	0	1	1
R3	0	1	0	1
R4	0	0	1	0
R5	1	0	1	0
R6	0	1	0	0
Perform a minhashing of the data, with the order of rows: R4, R6, R1, R3, R5, R2. Which of the following is the correct minhash value of the stated column? Note: we give the minhash value in terms of the original name of the row, rather than the order of the row in the permutation. These two schemes are equivalent, since we only care whether hash values for two columns are equal, not what their actual values are.

The minhash value for C1 is R4
The minhash value for C2 is R3
The minhash value for C3 is R2
The minhash value for C3 is R4
Question 3
Here is a matrix representing the signatures of seven columns, C1 through C7.
C1	C2	C3	C4	C5	C6	C7
1	2	1	1	2	5	4
2	3	4	2	3	2	2
3	1	2	3	1	3	2
4	1	3	1	2	4	4
5	2	5	1	1	5	1
6	1	6	4	1	1	4
Suppose we use locality-sensitive hashing with three bands of two rows each. Assume there are enough buckets available that the hash function for each band can be the identity function (i.e., columns hash to the same bucket if and only if they are identical in the band). Find all the candidate pairs, and then identify one of them in the list below.

C1 and C2
C4 and C5
C1 and C7
C1 and C3
Question 4
Find the set of 2-shingles for the "document":

ABRACADABRA
and also for the "document":


BRICABRAC
Answer the following questions:

How many 2-shingles does ABRACADABRA have?
How many 2-shingles does BRICABRAC have?
How many 2-shingles do they have in common?
What is the Jaccard similarity between the two documents"?
Then, find the true statement in the list below.

There are 8 shingles in common.
BRICABRAC has 8 2-shingles.
ABRACADABRA has 5 2-shingles.
There are 5 shingles in common.
Question 5
Here are eight strings that represent sets:
s1 = abcef
s2 = acdeg
s3 = bcdefg
s4 = adfg
s5 = bcdfgh
s6 = bceg
s7 = cdfg
s8 = abcd

Suppose our upper limit on Jaccard distance is 0.2, and we use the indexing scheme of Section 3.9.4 based on symbols appearing in the prefix (no position or length information). For each of s1, s3, and s6, determine how many other strings that string will be compared with, if it is used as the probe string. Then, identify the true count from the list below.

s6 is compared with 6 other strings.
s1 is compared with 7 other strings.
s6 is compared with 5 other strings.
s1 is compared with 6 other strings.
Question 6
Suppose we want to assign points to whichever of the points (0,0) or (100,40) is nearer. Depending on whether we use the L1 or L2 norm, a point (x,y) could be clustered with a different one of these two points. For this problem, you should work out the conditions under which a point will be assigned to (0,0) when the L1 norm is used, but assigned to (100,40) when the L2 norm is used. Identify one of those points from the list below.
(53,15)
(61,10)
(53,10)
(66,5)
In accordance with the Coursera Honor Code, I (Kevin O'Brien) certify that the answers here are my own work.
       
You cannot submit your work until you agree to the Honor Code. Thanks!
